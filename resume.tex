% !TEX program = xelatex

\documentclass{resume}
\usepackage{graphicx}
\usepackage{tabu}
\usepackage{multirow}
\usepackage{progressbar}

\begin{document}
\pagenumbering{gobble} % suppress displaying page number

\name{Cheng Peng}

\basicInfo{
	\email{pengcheng2@shanghaitech.edu.cn} \textperiodcentered\
	\phone{(+86) 151-0154-2294} \textperiodcentered\
	\github[hehelego]{https://github.com/hehelego}}

\section{\faGraduationCap\ Education}
\datedsubsection{\textbf{ShanghaiTech University}, Shanghai, China}{2020 -- Present}
\textit{Bachelor of Engineering} in Computer Science. Expected graduation date: June 2024\par
\begin{itemize}
	\item Overall GPA 3.89/4, ranked 2/176
	\item Major GPA 4/4.
	\item CS major courses taken:
	      \begin{description}
		      \item[CS100] Introduction to computer programming, A+
		      \item[CS101] Data Structure and Algorithm, A+
		      \item[CS110] Computer Architecture, A+
		      \item[CS120] Computer Network, A+
		      \item[CS121] Parallel Computing, A
		      \item[CS130] Operating System, A+
		      \item[CS240] Algorithm Design and Analysis, A+
		      \item[CS244] Theory of Computation, A
	      \end{description}
\end{itemize}

\section{\faChalkboardTeacher\ TA Experiences}

\datedsubsection{\textbf{CS101 Data Structure and Algorithm} at ShanghaiTech SIST, Shanghai, China}{Sep. 2022 -- Jan. 2023}
\role{Teaching Assistant}{Leader of the programming assignments group.}
\begin{itemize}
	\item In charge of setting four programming assignments.
	\item Maintain online judge system for the programming tasks.
	\item Carry out weekly recitation.
	\item Problem setter of homework covering asymptotic analysis, disjoint set union and NP-completeness.
\end{itemize}

\datedsubsection{\textbf{CS110 Computer Architecture} at ShanghaiTech SIST, Shanghai, China}{Feb. 2023 -- June. 2023}
\role{Teaching Assistant}{of the computer architecture course in the semester of Spring 2023}
\begin{itemize}
	\item Task setter of two programming homework.
	\item Design a project on performance optimization (discover bottleneck with profiler, improving cache hit rate, leveraging OpenMP thread parallelism and utilizing SIMD vectorization).
	\item Hold lab session weekly.
\end{itemize}

\section{\faNetworkWired\ Projects}

\datedsubsection{\textbf{Parallel K-means in C++}}{Apr. 2021 -- Jun. 2021}
\role{Programmer}{Individual course project of CS100 Introduction to Programming.}
\begin{itemize}
	\item Data parallel with C++ \texttt{std::thread}.
	\item Manually vectorization with Intel AVX2 SIMD instructions.
	\item Improve cache hit rate by transposing the access pattern.
\end{itemize}

\datedsubsection{\textbf{PintOS}}{Sep. 2022 -- Jan. 2023}
\role{Programmer}{Course project of CS130 Operating System. Collaborated with Huizhe Su, my fiancee.}
\begin{itemize}
	\item Implementing Thread scheduling and synchronization primitives.
	\item Implementing User program loading and argument passing.
	\item Implementing various system call handlers.
	\item Implementing page table management and swapping.
	\item Implementing basic file system with ext2 alike structure.
	\item Pass all the test case.
	\item Write clear and concise design document.
\end{itemize}

\datedsubsection{\textbf{Athernet}}{Sep. 2022 -- Jan. 2023}
\role{Architect and Programmer}{Course project of CS120 Computer Network. In collaboration with beloved Huizhe Su.}
\begin{itemize}
	\item Well-organized with \texttt{cargo-workspace}. Written in pure rust.
	\item Physics layer that send/receive chunk of data over acoustic channel either wireless or cabal-connected.
	\item MAC sublayer with CSMA/CD enabling bidirectional point-to-point reliable data transmission.
	\item IP protocol for network layer. A gateway with NAT. ICMP ping request and response.
	\item Basic TCP and UDP protocol support. Providing socket API that resembles rust \texttt{std::net::TcpStream}.
	\item FTP client with limited functionalities on Athernet.
	\item SOCKS5 proxy server for tunneling normal network traffic through Athernet.
\end{itemize}

\section{\faCogs\ Skills}
\begin{itemize}[parsep=0.5ex]
	\item English Proficiency:
	      \begin{itemize}
		      \item \textbf{TOEFL 103} (R 29, L 29, S 23, W 22).
		      \item Most of the course I've taken are taught in English.
		      \item English working environment in my previous TA experiences.
	      \end{itemize}
	\item Programming Skills:
	      \begin{itemize}
			  \item Express thoughts fluently with \textbf{Rust} and \textbf{Python}.
			  \item Proficient in \textbf{C++} and \textbf{C}.
			  \item Basic knowledge in \textbf{Haskell} as well as \textbf{Scala}.
	      \end{itemize}
	\item Developing Environment:
	      \begin{itemize}
		      \item Using \textbf{Arch Linux} on a daily basis for more than 2 years.
		      \item Working with \textbf{i3wm}, \textbf{fish shell}, \textbf{neovim} and \textbf{tmux} efficiently at ease.
		      \item Good command of \textbf{git} version control system and \textbf{lazygit} TUI git client.
	      \end{itemize}
\end{itemize}

\section{\faHeart[regular]\ Honors and Awards}
Not interested in glorify myself, I never apply for a scholarship or award.

\section{\faInfo\ Miscellaneous}
\begin{itemize}[parsep=0.5ex]
	\item Github: \url{https://github.com/hehelego}
	\item Dotfiles: \url{https://github.com/hehelego/dotfiles}
	\item Personal notebook: \url{https://github.com/hehelego/WhyNotMarkdown}
\end{itemize}

\end{document}
